\documentclass[conference]{IEEEtran}
\usepackage{cite}
\usepackage{graphicx}
\usepackage{algorithmic}
\usepackage{textcomp}
\usepackage{xcolor}
\usepackage{amsmath, amssymb, amsfonts}
\usepackage{hyperref}


% Judul
\title{Implementasi Algoritma Dijkstra Dalam Menemukan Jarak Terdekat Dari Lokasi Pengguna Ke Tanaman Yang Di Tuju}

% Penulis
\author{\IEEEauthorblockN{Michelle Angelina}
\IEEEauthorblockA{\textit{School of Electrical Engineering and Informatics} \\
\textit{Institut Teknologi Bandung} \\
Bandung, Indonesia\\
Email: 18320007@std.stei.itb.ac.id}
}

% folder gambar
\graphicspath{{./gambar/}}

\begin{document}

\maketitle

	\begin{abstract}
		Kebun Raya Purwodadi dengan luas area sekitar 85
hektar ternyata kekurangan papan informasi yang menyebabkan
pengunjung kerap kali kebingungan dalam mencari lokasi tanaman tertentu. Paper ini bertujuan untuk membuat simulasi
dari algoritma yang dapat menentukan jarak terdekat antara
pengunjung (pengguna program) dengan lokasi tanaman yang
dituju. Algoritma yang digunakan adalah algoritma Dijkstra
yang beroperasi secara menyeluruh \textit{(greedy)} untuk menguji
seitap persimpangan \textit{(Vertex)} dan jalan \textit{(Edge)} pada Kebun
Raya Purwodadi. Berdasarkan hasil simulasi dan pengujian,
kompleksitas ruang dari program ini adalah O(V) karena adanya
pembentukan array yang berisi V \textit{nodes} untuk mencari \textit{heap} minimum. Sementara, kompleksitas waktu dari algoritma tersebut
adalah O(V2).
	\end{abstract}

	\begin{IEEEkeywords}
		Dijkstra, \textit{Vertex}, \textit{Edge}, Tanaman.
	\end{IEEEkeywords} 

	\section{Pendahuluan}
	Studi mengenai penggunaan algoritma Dijkstra dalam mencari jarak terdekat dapat diimplementasikan pada kasus pencarian tanaman pada Kebun Raya Purwodadi seperti yang telah
dilakukan oleh Yusuf et al di tahun 2017\cite{implementasi}. Paper ini bertujuan untuk melakukan simulasi kembali terhadap penelitian
yang telah dilakukan dengan bahasa C serta mengevaluasi
efisiensinya melalui perhitungan kompleksitas waktu dan ruang dengan analisis Big-O.

Di Kecamatan Purwodadi, Kabupaten Pasuruan, terdapat
salah satu kebun raya di Indonesia yang bernama Kebun
Raya Purwodadi yang memiliki luas area hingga 85 hektar.
Kebun raya sebagai fasilitas rekreasi dan penelitian ini ternyata
kekurangan papan informasi yang seharusnya disediakan oleh
pihak pengelola. Hal ini menyebabkan banyaknya pengunjung
yang merasa kebingungan untuk mencari lokasi dari tanaman
tertentu. Oleh karena itu, Yusuf et al (2017) memutuskan
untuk membuat suatu aplikasi dengan memanfaatkan algoritma
Dijkstra untuk membantu pengunjung Kebun Raya Purwodadi
dalam mencari lokasi tertentu.

Algoritma Dijkstra digunakan karena algoritma ini dapat
beroperasi secara menyeluruh (algoritma \textit{greedy}) terhadap
semua alternatif fungsi serta durasi eksekusi yang lebih cepat
jika dibandingkan dengan algoritma serupa, yaitu BellmanFord. Algoritma ini akan mencari jalur dengan ’biaya’ atau
cost terendah antara dua titik dengan membandingkan semua
alternatif yang ada.

Pada kasus ini, masing-masing persimpangan di Kebun
Raya Purwodadi direpresentasikan sebagai \textit{vertex} dan setiap
jalan direpresentasikan sebagai \textit{edge}. Rute terdekat yang didapatkan akan diperoleh dari pembobotan setiap \textit{vertex} dan \textit{edge}
berdasarkan jarak antara titik pengguna dengan titik tujuan
atau tanaman.

	\section{Studi Pustaka}
	\subsection{Algoritma Dijkstra}
% ada algoritma
Algoritma Dijkstra adalah algoritma yang digunakan untuk
menemukan jarak jalur terpendek antara dua \textit{vertice} pada
graph berbobot dan tidak berarah sederhana~\cite{ref2}. Berbobot
berarti grafik memiliki \textit{edge} dengan suatu ’bobot’ atau harga.
Bobot dapat merepresentasikan jarak, waktu, atau apapun
yang memodelkan koneksi antara kedua \textit{node}. Tidak berarah
memiliki arti bahwa untuk setiap \textit{node} yang terhubung, kita
dapat mendekati suatu \textit{node} dari kedua arah. Pendekatan Dijikstra juga memiliki asumsi bahwa bobot pada \textit{edge} memiliki
nilai yang tidak negatif. Hal ini karena nilai bobot akan
terus dibandingkan dan diambil nilai yang paling kecil. Ada
banyak varian pada algoritma ini, namun pada percobaan
ini digunakan varian dimana suatu \textit{node} ditetapkan menjadi
\textit{source node}. Dari node inilah akan dicari jarak terpendek
diantara \textit{node} lain. Algoritma ini dicetuskan oleh Edsger
Wybe Dijkstra, salah seorang tokoh ternama di bidang \textit{computer science}~\cite{ref3}. Kompleksitas dari algoritma dijkstra adalah O(\begin{math} n^2 \end{math}), dengan \begin{math} n \end{math} menyatakan jumlah \textit{vertice} dari \textit{graph} yang
bersangkutan.


	\subsection{Kebun Raya Purwodadi}
Kebun Raya Purwodadi adalah kebun penelitian di Kecamatan Purwodadi, Jawa Timur. Ia juga dikenal dengan nama
Hortus Ilkim Kering Purwodadi dan didirikan tanggal 30 Januari 1941 oleh Dr. L.G.M. Baas Becking. Sebagai cabang dari
Kebun Raya Bogor, ia memiliki fungsi mengkoleksi tumbuhan
yang hidup di dataran rendah kering. Sebagai Balai Konservasi
Tumbuhan di bawah Pusat Konservasi Tumbuhan Kebun Raya,
Kedeputian Bidang Ilmu Pengetahuan Hayati LIPI, kebun raya
ini memiliki banyak tumbuhan yang dinaunginya. Dengan
menggunakan algoritma Dijkstra, diharapkan ia dapat membantu pengunjung mencari tanaman tertentu maupun jarak
yang paling optimal.

	\section{Metodologi Penelitian}
Peneliti menggunakan beberapa tahap dalam penyusunan
paper ini. Pertama, dilakukan pengkajian dan studi literatur
dengan membaca referensi paper yang berkaitan dan memilih
paper yang dapat menjadi acuan dalam penelitian yang dilakukan, sehingga dari pilihan topik dan tema yang berkaitan
secara luas dapat dikecilkan menjadi sebuah paper yang mencakup mayoritas dari topik yang dibahas. Setelah ditemukan
beberapa paper, dilakukan perangkuman untuk menentukan
paper yang sesuai sekaligus membahas poin-poin penting
dari paper yang ingin dicapai. Setelah kedua tahap tersebut
dilewati, penentuan paper yang dijadikan prototype penelitian
merupakan hal yang mudah dan menjadi titik pencapaian
dalam studi literatur dan pemilihan topik dari prototype penelitian yang dilakukan.

Setelah itu, tahap selanjutnya yang dilakukan oleh peneliti
adalah pembuatan prototype berupa program yang ditulis
dalam bahasa C. Pembuatan prototype berupa kode ini dilakukan terus-menerus dengan menggunakan metode trial and
error sehingga perlu dilakukan revisi hingga protoype kode
yang dibuat dapat mendapatkan output yang optimal dan
sesuai dengan spesifikasi yang diharapakan. Tahap terakhir
dari penelitian adalah pemaparan kode yang berhasil dijalankan tersebut ke dalam paper.
%ada gambar

	\section{Implementasi dan Pengujian}
	\subsection{Implementasi Graph pada Array dalam Bahasa C}
Program akan dimulai dengan pembacaan file bernama
\textit{listtanaman.txt}. File tersebut akan menyimpan informasi mengenai semua nama tanaman yang bersangkutan. Setelah pembacaan tersebut, akan dicari informasi mengenai bobot graph
yang menghubungkan \textit{node}. Informasi ini disimpan di dalam
matriks segitiga bawah kiri didalam file \textit{jarakantarpohon.txt}
yang juga dibuka saat program dijalankan. Matriks menggambarkan bobot antara jarak dua \textit{node} tanaman sekali saja karena
pemodelan \textit{undirected graph} yang memiliki jarak sama baik
dari \textit{a} ke \textit{b} maupun \textit{b} ke \textit{a}. Nilai \begin{math} -1 \end{math} akan menggambarkan
bagian \textit{node} yang tidak terhubung sama sekali dalam graph
dan juga dinyatakan dalam suatu variabel bernama int max
(Jaraknya sebesar tak hingga). Nilai jarak terpendek akan
disimpan dalam array tersebut selagi program berjalan.

	\subsection{Implementasi Algoritma Dijkstra dalam Bahasa C}
Dalam implementasi algoritma, abstraksi dengan menggunakan pseudocode dapat dibagi menjadi dua buah fungsi dan
satu program utama. Fungsi yang digunakan adalah fungsi
printgraph (Fungsi Graph) untuk memunculkan graph berukuran \begin{math} n \times n \end{math} ke layar pengguna. Algoritma dari fungsi tersebut
dapat dilihat pada bagian di bawah ini:
% ada algoritma

Fungsi kedua yang digunakan adalah fungsi pencari indeks
pada array yang akan diproses dengan menggunakan pendekatan algoritma Dijkstra. Abstraksi fungsi yang digunakan
dapat dilihat pada bagian berikut ini:
%ada algoritma

Program utama akan membaca file database tanaman
beserta jarak masing-masing tanaman dan akan mencetak
daftar tanaman yang berada di Kebun Raya Purwodadi.
Kemudian, program akan menerima input salah satu tanaman
terdekat dari pengguna sebagai penanda posisi awal pengguna.
Setelah itu, program akan kembali menerima input posisi
tanaman tujuan dan memproses pencarian rute terdekat dengan
algoritma Dijkstra. Rute yang diperlukan akan ditampilkan
dalam bentuk list nama tanaman yang harus dilalui pengguna
dan menampilkan jarak antara kedua tanaman tersebut.
Implementasi algoritma dalam abstraksi tersebut dapat dilihat
pada gambar di bawah ini:
%ada algoritma

Setelah pembacaan jumlah tanaman dari file, maka diperlukan graph atau jarak antar tanaman yang akan menjadi dasar
perhitungan dari pencarian rute terdekat. Proses memasukkan
graph dapat dilihat pada algoritma berikut ini:
%ada algoritma

Setelah data yang dibutuhkan dimasukkan, implementasi
dari algoritma Dijkstra untuk pencarian rute terdekat adalah
sebagai berikut:
%ada algoritma

	\subsection{Implementasi Program dalam Bahasa C}
	Implementasi program dalam bahasa C dapat dilihat
pada repository berikut. \url{https://github.com/ReynaldoAverill/Tugas7PMC}

	\subsection{Perhitungan Kompleksitas Waktu}
Kompleksitas dari program ini dengan notasi kompleksitas
Big O adalah O(\begin{math} n^2 \end{math}). Hal tersebut disebabkan pada loop
program bagian \textit{for}, terdapat loop \textit{for} lain yang berjumlah
dua loop (Terletak pada bagian \textit{assign} kondisi awal dan ketika
program menjalankan algoritma Djikstra). Karena hal tersebut,
akibatnya adalah kompleksitas waktu akan naik seiring dengan
naiknya \begin{math} n \end{math} program yang dijalankan, namun tidak bersifat
linear sehingga kompleksitas waktunya adalah O(\begin{math} n^2 \end{math}). Grafik
kompleksitas waktu dapat direpresentasikan pada gambar 1.

	\subsection{ Perhitungan Kompleksitas Tempat}
Matriks penyimpanan yang digunakan pada program ini
memiliki ukuran terbesar \begin{math} n \times n \end{math}, dengan nilai \begin{math} n \end{math} merepresentasikan banyak tanaman dalam file \textit{listtanaman.txt}. Program
akan melalui grafik dan menyimpan nilai bobot antara \textit{node}
sebesar matriks di atas, mengakibatkan program dengan kompleksitas O(\begin{math} n^2 \end{math}). Hal ini dapat dilihat pada grafik kompleksitas
tempat di gambar 2.

	\section{Kesimpulan}
Pada perhitungan Jarak Terdekat dalam suatu lokasi atau ruang dapat diimplementasikan penggunaan Algoritma Djikstra
dalam perhitungannya untuk mencapai suatu target pada ruang
tersebut dari suatu titik. Terbukti dari penelitian Kebun Raya
Purwodadi untuk menentukan Tanaman yang ingin dituju.


	\bibliographystyle{IEEEtran}
	\bibliography{reference.bib}
	

\end{document}
